% Informe técnico (Máximo 20 páginas)
% ===============
\section{Reporte técnico}

% Identificación y contextualización
% ----------------------------------
%   Contextualiza, en no más de una plana, el lugar de práctica
%         eg:
%            - Características de la empresa
%            - Área
%            - Tamaño de la empresa en relación al rubro
%            - etc
%\subsection{La empresa}
%\paragraph{}



% Generación y justificación
% --------------------------
%   Identifica el objetivo o interés de la empresa a ser logrado durante
%         la práctica y plantea un problema o pregunta a ser resuelto.

%  Propone una metodología, herramientas y/o modelos para el análisis
%         diseño, desarrollo y solución del objetivo/problema (utiliza supuestos
%         si no fueron aplicados durante el periodo de práctica).

%  Describe los análisis, mediciones o aplicaciones y los softwares
%         necesarios para la resolución del problema/demanda (utiliza supuestos
%         si no fueron aplicados durante el periodo de práctica).

%  Describe los resultados obtenidos apoyándose en indicadores concretos
%         o mediciones reales o estimadas.
%\subsection{Los objetivos}
%\paragraph{}



% Conclusión
% ----------
%  Concluye los resultados, incluyendo: limitaciones y relevancia de los
%         análisis obtenidos; las implicancias futuras de las decisiones tomadas
%         y se discuten eventuales modificaciones y pasos a seguir.
%\subsection{Conclusión}
